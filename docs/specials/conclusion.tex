\thispagestyle{empty}

\chapter*{Conclusion générale}

Notre projet intitulé \textbf{« Conception et développement d’une plateforme E-learning intégrant un modèle intelligent d’élucidation des documents »} synthétise le travail de quatre mois dans la société \textbf{« Globe Services Informatique »}. Il s’agit d’une plateforme appelée \textbf{« Undrstnd »} qui vise à améliorer l’expérience éducative des étudiants.\\

Notre rapport comporte cinq chapitres : Le premier chapitre a été consacré au contexte général du projet, l’étude de l’existant et la planification de notre travail selon la méthode 2TUP que nous avons adopté pour mettre en place notre solution. Le deuxième chapitre a abordé les concepts de base liée à notre travail, en explorant des notions clés de l’intelligence artificielle et du cloud computing. Le troisième chapitre a été consacré à l’analyse et la spécification des besoins fonctionnels et non fonctionnels via les diagrammes de cas d’utilisations et les diagrammes de séquence d’analyse. Dans le quatrième chapitre, nous avons présenté le modèle architectural de notre système, la conception de la base de données, la conception logicielle détaillée et le maquettage de quelques interfaces. Finalement, nous avons clôturé notre rapport par le cinquième chapitre qui a été consacré à la spécification des différents outils matériels et logiciels utilisés dans le développement de notre plateforme, à l’intégration et l’évaluation du modèle utilisé et nous avons fini par quelques captures d’écran du travail réalisé.\\

Ce travail nous a permis de mettre en œuvre les acquis théoriques que nous avons appris tout le long de notre cursus universitaire, de consolider et d’approfondir nos connaissances dans les bonnes pratiques de la gestion de projet.