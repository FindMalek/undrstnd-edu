%\thispagestyle{empty}
\thispagestyle{plain} 
\mtcaddchapter[Introduction générale]
\begin{center}
  \textbf{\Huge Introduction générale}
\end{center}

\noindent L'internet a, profondément, transformé les modes d'apprentissage et de consommation des informations éducatives. Ces dernières décennies, l'éducation à distance et l'accès aux ressources pédagogiques en ligne ont connu une croissance exponentielle. En effet, la demande pour des solutions E-learning efficaces est en forte croissance, notamment avec l'émergence de technologies éducatives avancées. L’intelligence artificielle est désormais omniprésente. Elle joue un rôle essentiel dans l'amélioration de l'expérience d'apprentissage en évaluant les demandes des étudiants et en fournissant des réponses précises, menant à une meilleure compréhension des sujets étudiés.\\
\noindent Néanmoins, de nombreux défis persistent, tels que la personnalisation de l'apprentissage, l'intégration de ressources pédagogiques variées et la création d'une expérience utilisateur enrichissante. \\
\noindent Face à ces défis, nous avons puisé notre inspiration dans notre propre expérience de la vie estudiantine pour proposer une idée d’un projet de fin d’études. Son objectif principal est de révolutionner l'expérience d'apprentissage des étudiants et des enseignants. Ainsi, nous cherchons à concevoir et développer une plateforme e-learning conviviale et gratuite intégrant un modèle intéractif basé sur un algorithme intelligent. Ce modèle doit être capable d'élucider, parfaitement, les requêtes des étudiants et de générer une réponse cohérente qui se focalise sur un document partagé par l’étudiant, l’enseignant ou sur des ressources externes. \\
\noindent Notre rapport contiendra cinq chapitres qui se présenteront comme suit: \\
\textbf{Le premier chapitre} présentera le cadre général du projet, l'étude de l'existant et la planification de notre travail selon la méthode 2TUP.\\
\textbf{Le deuxième chapitre} abordera les concepts de base liés à notre travail, en explorant des notions clés de l'intelligence artificielle et du cloud computing.\\
\textbf{Le troisième chapitre} sera consacré à l'analyse et à la spécification des besoins fonctionnels et non fonctionnels, illustrés par des diagrammes de cas d'utilisation et des diagrammes de séquence d'analyse.\\
\textbf{Le quatrième chapitre} présentera le modèle architectural de notre système, la conception de la base de données, la conception logicielle détaillée et le maquettage de quelques interfaces. Enfin, \textbf{le cinquième chapitre} sera consacré à la spécification des différents outils matériels et logiciels utilisés dans le développement de notre plateforme, à l'intégration et à l'évaluation du modèle d’apprentissage automatique utilisé, ainsi qu'à quelques captures d'écran du travail réalisé.